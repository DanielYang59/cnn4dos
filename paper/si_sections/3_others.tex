% Supporting Information Section Three: Other Additional Information

\section{Other additional information}

\subsection{Additional information on machine learning environment}

The hyperparameter optimization for the proposed CNN model was performed using NVIDIA\textsuperscript{\textregistered} V100 Tensor Core GPUs,
courtesy of the Australian National Computational Infrastructure's (NCI) HPC Gadi system.
The HyperBand algorithm \cite{li2018hyperband}, as implemented in KerasTuner \cite{omalley2019kerastuner} , facilitated the tuning process.

% Supplementary Table 19
\begin{table}[h]
  \centering
  \begin{tabular}{llll}
    \hline
    Name                & Version & Build          & Channel     \\
    \hline
    conda               & 4.13.0  & py39h06a4308_0 & anaconda    \\
    cuda-nvcc           & 11.8.89 & 0              & nvidia      \\
    cuda-toolkit        & 11.8.0  & 0              & nvidia      \\
    cudatoolkit         & 11.6.0  & habf752d_9     & nvidia      \\
    ipykernel           & 6.9.1   & py39h06a4308_0 & anaconda    \\
    ipython             & 7.34.0  & pypi_0         & pypi        \\
    jupyter\_client     & 7.2.2   & py39h06a4308_0 & anaconda    \\
    jupyterlab          & 3.4.4   & py39h06a4308_0 & anaconda    \\
    keras               & 2.9.0   & pypi_0         & pypi        \\
    keras-preprocessing & 1.1.2   & pypi_0         & pypi        \\
    keras-tuner         & 1.3.5   &                & pypi        \\
    matplotlib          & 3.5.1   & py39h06a4308_1 & anaconda    \\
    numpy               & 1.22.4  & pypi_0         & pypi        \\
    pandas              & 1.3.4   & py39h8c16a72_0 & anaconda    \\
    python              & 3.9.12  & h12debd9_1     & anaconda    \\
    scikit-learn        & 1.1.1   & py39h6a678d5_0 & anaconda    \\
    scipy               & 1.8.1   & py39hddc5342_3 & conda-forge \\
    tensorboard         & 2.9.1   & pypi_0         & pypi        \\
    tensorflow          & 2.9.3   & pypi_0         & pypi        \\
    tensorflow-gpu      & 2.9.3   & pypi_0         & pypi        \\
    xlrd                & 2.0.1   & pyhd3eb1b0_0   & anaconda    \\
    yaml                & 0.2.5   & h7b6447c_0     & anaconda    \\
    \hline
  \end{tabular}
  \caption{Core packages in the deep learning environment}
  \tablefootnote{The complete list of packages is provided along with the source code.}
  \label{si_table19}
\end{table}


\subsection{Additional information on data analysis and visualization environment}
Data analysis and visualization tasks were executed on an Apple MacBook Air featuring an octa-core M2 chip.
For accuracy, the TensorFlow-Metal plugin for Mac-based GPU acceleration was deliberately excluded,
as it could potentially yield erroneous CNN model predictions.

The illustrations of CNN architecture displayed in Figure 1 of the main text,
as well as  Supplementary Figure 21 and Supplementary Figure 30,
were obtained from the GitHub repository dair-ai/ml-visuals \cite{Saravia_ML_Visuals_2021}.
Blender was used to create and render the visual representation of the catalyst analysis pipeline shown in Figure 1 of the main text.

% Supplementary Table 20
\begin{table}[h]
  \centering
  \begin{tabular}{llll}
    \hline
    Name             & Version & Build      & Channel \\
    \hline
    bokeh            & 3.2.2   & pypi_0     & pypi    \\
    colorcet         & 3.0.1   & pypi_0     & pypi    \\
    heatmapz         & 0.0.4   & pypi_0     & pypi    \\
    keras            & 2.13.1  & pypi_0     & pypi    \\
    keras-tuner      & 1.3.5   & pypi_0     & pypi    \\
    matplotlib       & 3.7.2   & pypi_0     & pypi    \\
    numpy            & 1.24.3  & pypi_0     & pypi    \\
    pandas           & 2.0.3   & pypi_0     & pypi    \\
    python           & 3.11.4  & hb885b13_0 &         \\
    pyyaml           & 6.0.1   & pypi_0     & pypi    \\
    scikit-learn     & 1.3.0   & pypi_0     & pypi    \\
    scipy            & 1.11.1  & pypi_0     & pypi    \\
    seaborn          & 0.12.2  & pypi_0     & pypi    \\
    tensorboard      & 2.13.0  & pypi_0     & pypi    \\
    tensorflow       & 2.13.0  & pypi_0     & pypi    \\
    tensorflow-macos & 2.13.0  & pypi_0     & pypi    \\
    \hline
  \end{tabular}
  \caption{Core packages in the data analysis and visualization environment}
  \tablefootnote{The complete list of packages is provided along with the source code.}
  \label{si_table20}
\end{table}
