Section 2 Additional details on machine learning method

Section 2.1 d-band center and adsorption energy relations


Supplementary Figure 18 | Scatter plot of d-Band Center and Adsorption Energy Relationships. This figure delineates the relationship between d-band center and adsorption energies for various species: (a) CO2, (b) COOH, (c) CO, (d) CHO, (e) CH2O, (f) OCH3, (g) O, and (h) OH. Data from all six substrates investigated in this study are included.


Section 2.2 Feature correlation analysis


Supplementary Figure 19 | Pairwise Plot of Adsorption Energy and Key Descriptors. This chart displays the pairwise relationships between CO adsorption energy and its four most correlated descriptors–d-band center (spin-up) δεd↑, lattice parameter γ, vacuum level Evac, and electronegativity on the Allen scale XAllen –as identified by Kendall rank correlation analysis. Data from all six substrates investigated in this study are incorporated.


Supplementary Table 15
Notations for elementary descriptors and electronic descriptors
Elementary Descriptors	Electronic Descriptors
R	atomic radius	α, β, γ	lattice parameters
Eea	electron affinity	ΦDFT	DFT calculated work function
Φ	work function	Evac	vacuum level
XAllen, ΧP, ΧRevP	electronegativity in Allen, Pauling and revised Pauling scales	Eg, Eg↑, Eg↓	average, spin-up and spin-down band gaps
Ar	relative atomic mass	ed	number of d-electrons
Ei	ionisation energy	δεd↑	d-band centre (spin-up)
G	group number	Wd	d-band width
P	period number	eBader	Bader charge
V	number of valence electrons


Section 2.3 Input pipeline construction
The format of the 2D array for the eDOS is naturally well-suited for use as neural network inputs. In this study, the array adopts a shape of [numSamplings, numOrbitals, numChannels], as illustrated in Supplementary Figure 20. Here, “numSamplings” represents the number of data points sampled across the energy spectrum, “numOrbitals” denotes the number of DOS orbitals, and “numChannels” indicates the number of atomic channels.


Supplementary Figure 20 | Schematic representation of the input eDOS data structure.

For practical purposes, certain spatial hierarchies are expected to remain invariant, such as the energy range over which the eDOS is sampled. In this work, we maintain a consistent energy range of -14 eV to 6 eV, sampled at a density of 50 points⋅eV-1. In addition to maintaining a fixed energy range, the following steps are taken to standardize the input eDOS data:
1.	For atoms lacking d-electrons, the eDOS is zero-padded.
2.	f-orbital eDOS data are excluded.
3.	The eDOS of adsorbates is zero-padded along the Channels axis to align with the adsorbate that contains the maximum number of atoms, which, in this study, is the *OCH3 species. This padded eDOS is then appended to the metal eDOS.

As a result of these preprocessing steps, the input eDOS arrays consistently have a shape of [4000, 9, 6]. This corresponds to 4000 sampling points across the energy range, 9 eDOS orbitals, and 6 atomic channels.


Section 2.4 Hyperparameter tuning for CNN

Supplementary Table 16
The hyperparameter searching space
Hyperparameter	Search space	Description
learningRate	[10-4, 10-2], logscale	Learning rate of Adam optimiser
dropoutRate	[0, 1]	Dropout rate
root_fc0	128, 256, 512	Output dimensionality of 1st fully connected layer in root
root_fc1	256, 512, 1024, 2048	Output dimensionality of 2nd fully connected layer in root
root_act	"tanh", "relu", "sigmoid"	Activation function of fully connected layers in root
kernelSize	[2, 32]	Width of convolution window
numFilters	[8, 64]	Number of filters in the convolution
numConvLayers	[1, 16]	Number of convolution layers
numConvBlocks	[1, 16]	Number of convolution blocks
br_fc0	16, 32, 64, 128, 256, 512	Output dimensionality of 1st fully connected layer in branch
br_fc1	8, 16, 32, 64, 128, 256	Output dimensionality of 2nd fully connected layer in branch
br_act	"tanh", "relu", "sigmoid"	Activation function of fully connected layers in branch


Supplementary Table 17
Optimal hyperparameters identified with Hyperband algorithm
Hyperparameter	Optimal value	Description
learningRate	10-3	Learning rate of Adam optimiser
dropoutRate	0.3	Dropout rate
root_fc0	256	Output dimensionality of 1st fully connected layer in root
root_fc1	512	Output dimensionality of 2nd fully connected layer in root
root_act	“relu”	Activation function of fully connected layers in root
kernelSize	10	Width of convolution window
numFilters	17	Number of filters in the convolution
numConvLayers	3	Number of convolution layers
numConvBlocks	1	Number of convolution blocks
br_fc0	256	Output dimensionality of 1st fully connected layer in branch
br_fc1	128	Output dimensionality of 2nd fully connected layer in branch
br_act	“tanh”	Activation function of fully connected layers in branch


Supplementary Table 18
Prediction mean absolute errors of the CNN model
Adsorbate	Original dataset (eV)	Augmented dataset (eV)
CO2	0.0447	0.0336
COOH	0.0436	0.0581
CO	0.0402	0.0402
CHO	0.0614	0.0778
CH2O	0.0553	0.0632
OCH3	0.0605	0.0636
O	0.1135	0.1081
OH	0.0547	0.0602
H	0.0703	0.0646
Note. The same CNN model, trained on the augmented dataset, was used for all predictions. The term “Original dataset” refers to evaluations conducted on the original dataset without any augmented data; it does not imply the use of a different CNN model trained solely on the original dataset. Similarly, “Augmented dataset” refers to evaluations on the dataset with both original data and augment data, not a distinct model.


Section 2.5 Occlusion experiment


Supplementary Figure 21 | Illustration of the occlusion experiment. This figure demonstrates the step-by-step process of the occlusion experiment. The input eDOS array first undergoes zero-padded, and then a masker of width “width” is applied, resulting in an occlusion array that preserves the shape of the input array.


Supplementary Figure 22 | Occlusion Experiment with a masker width of “1”. This figure shows an occlusion experiment on the Ge atom in the initial state of CO adsorption on Ge@g-C3N4.


Supplementary Figure 23 | Occlusion Experiment with a masker width of “3”. This figure shows an occlusion experiment on the Ge atom in the initial state of CO adsorption on Ge@g-C3N4.


Supplementary Figure 24 | Occlusion Experiment with a masker width of “5”. This figure shows an occlusion experiment on the Ge atom in the initial state of CO adsorption on Ge@g-C3N4.

Supplementary Figure 25 | Occlusion Experiment with a masker width of “11”. This figure shows an occlusion experiment on the Ge atom in the initial state of CO adsorption on Ge@g-C3N4.


Supplementary Figure 26 | Occlusion Experiment with a masker width of “21”. This figure shows an occlusion experiment on the Ge atom in the initial state of CO adsorption on Ge@g-C3N4.

Supplementary Figure 27 | Occlusion Experiment with a masker width of “31”. This figure shows an occlusion experiment on the Ge atom in the initial state of CO adsorption on Ge@g-C3N4.


Supplementary Figure 28 | Occlusion Experiment with a masker width of “41”. This figure shows an occlusion experiment on the Ge atom in the initial state of CO adsorption on Ge@g-C3N4.

Supplementary Figure 29 | Occlusion Experiment with a masker width of “51”. This figure shows an occlusion experiment on the Ge atom in the initial state of CO adsorption on Ge@g-C3N4.


Section 2.6 Shifting experiment


Supplementary Figure 30 | Illustration of the shifting experiment. This figure illustrates the sequential stages of the shifting experiment. Initially, the input eDOS array is shifted along the Energy axis. Subsequently, it undergoes zero-padding and cropping, ensuring a consistent shape based on the shifting direction, to match the input eDOS array. The resulting shifted array is then processed by the CNN model to assess the disturbance in adsorption energy caused by the shifting operation.


Supplementary Figure 31 | Shifting Experiment on p orbital. Effect of orbital shifting on CO adsorption energy, predicted by the CNN model, for single metal atom catalysts supported on g-C3N4. The perturbations caused by shifting of entire p orbital are presented. The shifting step size corresponds to the energy resolution of the eDOS, set at 0.005 eV in this study. A positive eDOS shift indicates a shift towards higher energy levels, and vice versa.
