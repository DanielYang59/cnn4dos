% SI Section Three: Other Additional Information

\section{Other additional information}
\label{supp_sec4_env}

\subsection{Additional information on machine learning environment}
\label{supp_sec4.1_ml_env}

The hyperparameter optimization for the proposed CNN model was performed using NVIDIA\textsuperscript{\textregistered} V100 Tensor Core GPUs,
courtesy of the Australian National Computational Infrastructure's (NCI) HPC Gadi system.
The HyperBand algorithm \cite{li2018hyperband}, as implemented in
KerasTuner \cite{omalley2019kerastuner}, facilitated the tuning process.

% Supplementary Table 19: Packages in the DL env
\begin{table}[htbp]
\label{supp_table19:pack_dl_env}
  \caption{Core packages in the deep learning environment.}
  \small
  \center
  \begin{tabularx}{0.9\textwidth}{@{}l *{3}{X} @{}}
    \toprule
    Name                 & Version  & Build                         & Channel      \\
    \midrule
    conda                & 4.13.0   & py39h06a4308\textunderscore0  & anaconda     \\
    cuda-nvcc            & 11.8.89  & 0                             & nvidia       \\
    cuda-toolkit         & 11.8.0   & 0                             & nvidia       \\
    cudatoolkit          & 11.6.0   & habf752d\textunderscore9      & nvidia       \\
    ipykernel            & 6.9.1    & py39h06a4308\textunderscore0  & anaconda     \\
    ipython              & 7.34.0   & pypi\textunderscore0          & pypi         \\
    jupyter\_client      & 7.2.2    & py39h06a4308\textunderscore0  & anaconda     \\
    jupyterlab           & 3.4.4    & py39h06a4308\textunderscore0  & anaconda     \\
    keras                & 2.9.0    & pypi\textunderscore0          & pypi         \\
    keras-preprocessing  & 1.1.2    & pypi\textunderscore0          & pypi         \\
    keras-tuner          & 1.3.5    &                               & pypi         \\
    matplotlib           & 3.5.1    & py39h06a4308\textunderscore1  & anaconda     \\
    numpy                & 1.22.4   & pypi\textunderscore0          & pypi         \\
    pandas               & 1.3.4    & py39h8c16a72\textunderscore0  & anaconda     \\
    python               & 3.9.12   & h12debd9\textunderscore1      & anaconda     \\
    scikit-learn         & 1.1.1    & py39h6a678d5\textunderscore0  & anaconda     \\
    scipy                & 1.8.1    & py39hddc5342\textunderscore3  & conda-forge  \\
    tensorboard          & 2.9.1    & pypi\textunderscore0          & pypi         \\
    tensorflow           & 2.9.3    & pypi\textunderscore0          & pypi         \\
    tensorflow-gpu       & 2.9.3    & pypi\textunderscore0          & pypi         \\
    xlrd                 & 2.0.1    & pyhd3eb1b0\textunderscore0    & anaconda     \\
    yaml                 & 0.2.5    & h7b6447c\textunderscore0      & anaconda     \\
    \bottomrule
  \end{tabularx}

  \smallskip

  \begin{flushright}
  \begin{minipage}{\textwidth}
    \footnotesize\textit{Note:} The complete list of packages is
      provided along with the source code.
  \end{minipage}
  \end{flushright}
\end{table}

\subsection{Additional information on data analysis and visualization environment}
\label{supp_sec4.2_vis_env}

Data analysis and visualization tasks were executed on an Apple MacBook Air featuring an octa-core M2 chip.
For accuracy, the TensorFlow-Metal plugin for Mac-based GPU acceleration was deliberately excluded,
as it could potentially yield erroneous CNN model predictions.

The illustrations of CNN architecture displayed in \cref{main_fig1:pipeline} of the main text,
as well as \cref{supp_fig21:occlusion} and \cref{supp_fig30:shifting},
were obtained from the GitHub repository dair-ai/ml-visuals \cite{Saravia_ML_Visuals_2021}.
Blender was used to create and render the visual representation of the
catalyst analysis pipeline shown in \cref{main_fig1:pipeline} of the main text.

% Supplementary Table 20: Packages in data analysis and vis env
\begin{table}[htbp]
\label{supp_table20:pack_vis_env}
  \caption{Core packages in data analysis and visualization environment.}
  \small
  \center
  \begin{tabularx}{0.9\textwidth}{@{}l *{3}{X} @{}}
    \toprule
    Name              & Version  & Build                     & Channel  \\
    \midrule
    bokeh             & 3.2.2    & pypi\textunderscore0      & pypi     \\
    colorcet          & 3.0.1    & pypi\textunderscore0      & pypi     \\
    heatmapz          & 0.0.4    & pypi\textunderscore0      & pypi     \\
    keras             & 2.13.1   & pypi\textunderscore0      & pypi     \\
    keras-tuner       & 1.3.5    & pypi\textunderscore0      & pypi     \\
    matplotlib        & 3.7.2    & pypi\textunderscore0      & pypi     \\
    numpy             & 1.24.3   & pypi\textunderscore0      & pypi     \\
    pandas            & 2.0.3    & pypi\textunderscore0      & pypi     \\
    python            & 3.11.4   & hb885b13\textunderscore0  &          \\
    pyyaml            & 6.0.1    & pypi\textunderscore0      & pypi     \\
    scikit-learn      & 1.3.0    & pypi\textunderscore0      & pypi     \\
    scipy             & 1.11.1   & pypi\textunderscore0      & pypi     \\
    seaborn           & 0.12.2   & pypi\textunderscore0      & pypi     \\
    tensorboard       & 2.13.0   & pypi\textunderscore0      & pypi     \\
    tensorflow        & 2.13.0   & pypi\textunderscore0      & pypi     \\
    tensorflow-macos  & 2.13.0   & pypi\textunderscore0      & pypi     \\
    \bottomrule
  \end{tabularx}

  \smallskip

  \begin{flushright}
  \begin{minipage}{\textwidth}
    \footnotesize\textit{Note:} The complete list of packages
      is provided along with the source code.
  \end{minipage}
  \end{flushright}
\end{table}
