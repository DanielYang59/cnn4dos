% Abstract

\begin{abstract}
\label{main_sec_abstract}

Single-atom catalysts (SACs) have emerged as pivotal frontiers for catalyzing a myriad of chemical reactions,
yet the diverse combinations of active elements and support materials, the nature of coordination environments,
elude traditional methodologies in searching optimal SAC systems with superior catalytic performance.
Herein, by integrating multi-branch Convolutional Neural Network (CNN) analysis models
to hybrid descriptor based activity volcano plot,
two-dimensional (2D) SAC system composed of diverse metallic single atoms anchored on six type of 2D supports,
including graphitic carbon nitride (g-C\textsubscript{3}N\textsubscript{4}), nitrogen-doped graphene,
graphene with dual-vacancy, black phosphorous, boron nitride (BN), and C\textsubscript{2}N,
are screened for efficient CO\textsubscript{2}RR.
Starting from establishing a correlation map between the adsorption energies of
intermediates and diverse electronic and elementary descriptors,
sole singular descriptor lost magic to predict catalytic activity, including d-band center.
Deep learning method utilizing multi-branch CNN model therefore was employed,
using 2D electronic density of states (eDOS) as input to predict adsorption energies.
Hybrid-descriptor enveloping both C- and O-types of CO\textsubscript{2}RR intermediates
was introduced to construct volcano plots and limiting potential periodic table,
aiming for intuitive screening of catalyst candidates for efficient CO\textsubscript{2} reduction to CH\textsubscript{4}.
The eDOS occlusion experiments were performed to unravel individual orbital contribution to adsorption energy.
To explore the electronic scale principle governing practical engineering catalytic CO\textsubscript{2}RR activity,
orbitalwise eDOS shifting experiments based on CNN model were employed.
The study involves examining the adsorption energy and,
consequently, catalytic activities while varying supported single atoms.
This work offers a tangible framework to inform both theoretical screening and experimental synthesis,
thereby paving the way for systematically designing efficient SACs.

\end{abstract}
