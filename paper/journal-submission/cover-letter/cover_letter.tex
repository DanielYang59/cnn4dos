% Cover Letter
\documentclass[a4paper, 12pt]{article}


\usepackage{booktabs}
\usepackage{tabularx}


\begin{document}


\title{Cover Letter for: Convolutional Neural Networks and Volcano Plots:
    Screening and Prediction of Two-Dimensional Single-Atom Catalysts
    for CO\textsubscript{2} Reduction Reactions}

\date{\today}
\maketitle
\newpage


\section{Affiliation and contact information for the corresponding authors}

\begin{table}[htbp]
\label{cov_let_table:corr_author_contact}
    \begin{tabularx}{\textwidth}{p{2.5cm}p{6cm}l}
    \toprule
    Corresponding Author  & Affiliation                                             & Contact Email       \\
    \midrule
    Ting Liao             & School of Mechanical Medical and Process Engineering,
                            Queensland University of Technology, George Street,
                            Brisbane, QLD 4000, Australia                           & t3.liao@qut.edu.au   \\
    Ziqi Sun              & School of Chemistry and Physics,
                            Queensland University of Technology, George Street,
                            Brisbane, QLD 4000, Australia                           & ziqi.sun@qut.edu.au  \\
    \bottomrule
    \end{tabularx}
\end{table}


\section{Importance and appropriateness for submission}

\noindent \textbf{High Predictive Accuracy:} \\
    Our computational model demonstrates accuracy in predicting the adsorption energies
    of various CO2RR and HER adsorbates on single-atom catalysts (SACs),
    achieving a mean absolute error (MAE) at the 0.1 eV level.
    Our eDOS occlusion and shifting experiments elucidate the orbitalwise contributions
    to adsorption behavior and potential changes resulting from
    electronic density of states (eDOS) disturbances.
    The chemical meaningfulness of our predictions is validated through
    crystal orbital Hamilton population (COHP) analysis. \\

\noindent \textbf{Versatile Predictive Model:} \\
    Our convolutional neural network (CNN) model offers distinct advantages.
    Firstly, it does not rely on species-specific parameters,
    making it applicable beyond CO2RR to a wide range of electrochemical reduction reactions.
    Secondly, our model requires only the eDOS of supported single metal atoms and adsorbates,
    eliminating the need for substrate-specific parameters
    and significantly reducing memory footprint.
    Additionally, utilizing a multi-branched CNN architecture further reduced
    the model's memory footprint. \\

\noindent \textbf{Improved Linear Scaling Relations:} \\
    We present revised linear scaling relations that reduce scaling errors
    with limited computational overhead.
    This advancement paves the way for the generation of accurate volcano plots. \\

\noindent \textbf{Efficient Catalyst Screening Pipeline:} \\
    Our screening pipeline integrates volcano plots to predict directions
    for optimization, and eDOS shifting experiments to identify methods
    for achieving predicted volcano plot outcomes.


\section{Reviewers and exclusion of referees}

TODO: need supervisor input.

\section{Prior discussions with editor}

Authors declared no prior discussions with \textit{Nature Communications} editors about this work.


\end{document}
% End of Cover Letter
